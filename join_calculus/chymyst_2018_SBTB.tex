\batchmode
\makeatletter
\def\input@path{{/Users/sergei.winitzki/Code/talks/join_calculus/}}
\makeatother
\documentclass[english]{beamer}
\usepackage[T1]{fontenc}
\usepackage[latin9]{inputenc}
\setcounter{secnumdepth}{3}
\setcounter{tocdepth}{3}
\usepackage{babel}
\usepackage{array}
\usepackage{calc}
\usepackage{amstext}
\usepackage{graphicx}
\ifx\hypersetup\undefined
  \AtBeginDocument{%
    \hypersetup{unicode=true,pdfusetitle,
 bookmarks=true,bookmarksnumbered=false,bookmarksopen=false,
 breaklinks=false,pdfborder={0 0 1},backref=false,colorlinks=true}
  }
\else
  \hypersetup{unicode=true,pdfusetitle,
 bookmarks=true,bookmarksnumbered=false,bookmarksopen=false,
 breaklinks=false,pdfborder={0 0 1},backref=false,colorlinks=true}
\fi

\makeatletter

%%%%%%%%%%%%%%%%%%%%%%%%%%%%%% LyX specific LaTeX commands.
%% Because html converters don't know tabularnewline
\providecommand{\tabularnewline}{\\}

%%%%%%%%%%%%%%%%%%%%%%%%%%%%%% Textclass specific LaTeX commands.
% this default might be overridden by plain title style
\newcommand\makebeamertitle{\frame{\maketitle}}%
% (ERT) argument for the TOC
\AtBeginDocument{%
  \let\origtableofcontents=\tableofcontents
  \def\tableofcontents{\@ifnextchar[{\origtableofcontents}{\gobbletableofcontents}}
  \def\gobbletableofcontents#1{\origtableofcontents}
}
\newenvironment{lyxcode}
  {\par\begin{list}{}{
    \setlength{\rightmargin}{\leftmargin}
    \setlength{\listparindent}{0pt}% needed for AMS classes
    \raggedright
    \setlength{\itemsep}{0pt}
    \setlength{\parsep}{0pt}
    \normalfont\ttfamily}%
   \def\{{\char`\{}
   \def\}{\char`\}}
   \def\textasciitilde{\char`\~}
   \item[]}
  {\end{list}}

%%%%%%%%%%%%%%%%%%%%%%%%%%%%%% User specified LaTeX commands.
\usetheme[secheader]{Boadilla}
\usecolortheme{seahorse}
\author{Sergei Winitzki}
\date{November 18, 2018}
\setbeamertemplate{headline}{} % disable headline at top
\setbeamertemplate{navigation symbols}{} % disable navigation bar at bottom
\title[Declarative distributed concurrency]{Declarative distributed concurrency in Scala}
\institute[SBTB 2018]{Scale by the Bay 2018}

\makeatother

\begin{document}
\frame{\titlepage}
\begin{frame}{Talk summary}


\framesubtitle{How I learned to forget semaphores and to love concurrency}

\texttt{Chymyst} = an implementation of the Chemical Machine (CM)
paradigm
\begin{itemize}
\item A \emph{declarative language} for concurrent \& parallel computations
\begin{itemize}
\item largely unknown and unused by the software engineering community
\item available as an \href{https://github.com/Chymyst/chymyst-core}{open-source library \& embedded DSL}
for Scala
\item presented in my SBTB talks in 2016 and 2017
\end{itemize}
\item CM $\approx$ Actors made purely functional and auto-parallelized
\item Intuitions about why CM works better than other concurrency models
\begin{itemize}
\item Comparison with related work: ING Baker, BPMN (workflow)
\end{itemize}
\item New extension for distributed programming: DCM
\item Code examples and demos
\end{itemize}
Not in this talk: academic theory
\begin{itemize}
\item Petri nets, $\pi$-calculus, join calculus, joinads, mobile agent
calculus...
\item DCM formulated within some theory of distributed programming?
\end{itemize}
\end{frame}

\begin{frame}{Concurrent \& parallel programming: How we cope}

\emph{Imperative} concurrency \& parallelism is difficult to reason
about:
\begin{itemize}
\item low-level API: callbacks, threads, semaphores, mutex locks
\item hard to reason about mutable state and running processes
\item hard to test -- non-deterministic runtime behavior!
\begin{itemize}
\item race conditions, deadlocks, livelocks
\end{itemize}
\end{itemize}
\begin{center}
\vspace{-0.2cm}
Known declarative approaches to avoid these problems:\vspace{-0.5cm}
\par\end{center}

\begin{center}
\begin{tabular}{|>{\centering}p{0.35\textwidth}|c|>{\centering}p{0.3\textwidth}|}
\hline 
\textbf{\footnotesize{}Kind of concurrency} &
\textbf{\footnotesize{}Formal structure} &
\textbf{\footnotesize{}Scala code}\tabularnewline
\hline 
\hline 
synchronous parallelism &
applicative functor &
Spark, \texttt{\textcolor{blue}{\scriptsize{}.par.map()}}\tabularnewline
\hline 
asynchronous streaming DAG &
monadic functor &
\texttt{\textcolor{blue}{\scriptsize{}Future}}, \texttt{\textcolor{blue}{\scriptsize{}async}}/\texttt{\textcolor{blue}{\scriptsize{}await}},
\texttt{\footnotesize{}RxJava}, Akka Streams\tabularnewline
\hline 
unrestricted streaming &
recursive monad &
Flink, fs2, ZIO\tabularnewline
\hline 
unrestricted concurrency &
? &
Akka, \texttt{\footnotesize{}Chymyst}\tabularnewline
\hline 
\end{tabular}
\par\end{center}

For distributed computing: challenges remain
\begin{itemize}
\item coordination and consensus, persistence and fault tolerance
\item cluster configuration and discovery
\begin{itemize}
\item distributed coordination as a service: Apache ZooKeeper, \texttt{etcd}
\end{itemize}
\end{itemize}
\end{frame}

\begin{frame}{``Dining philosophers''}


\framesubtitle{The paradigmatic problem of concurrency, parallelism and resource
contention}

\href{https://en.wikipedia.org/wiki/Dining_philosophers_problem}{Five philosophers sit at a round table},
taking turns eating and thinking for random time intervals
\begin{center}
\includegraphics[height=4cm]{3_Users_sergei_win\includegraphics[height=4cm]{An_illustration_of_the_dining_philosophers_problem}
\par\end{center}

Problem: simulate the process, avoiding deadlock and starvation

Solutions: \href{https://rosettacode.org/wiki/Dining_philosophers}{Rosetta Code}
\begin{itemize}
\item Can this be implemented via functional streams? (I think not.)
\item The Chemical Machine code is purely declarative
\end{itemize}
\end{frame}

\begin{frame}{From Actors to the Chemical Machine}

Modify the Actor model by adding new requirements:
\begin{itemize}
\item when messages arrive, actors are auto-created, maybe \emph{in parallel}
\item actors may wait atomically for messages in \emph{several} different
mailboxes
\end{itemize}
It follows from these requirements that... 
\begin{itemize}
\item Auto-created actor instances are stateless and invisible to user
\item User code defines \emph{mailboxes} and \emph{computations} that consume
messages
\item Repeated messages may be consumed in parallel
\item Messages are sent to mailboxes, not to specific actor instances:
\end{itemize}
\begin{minipage}[c][1\totalheight][t]{0.5\columnwidth}%
\begin{lyxcode}
\textcolor{blue}{\scriptsize{}//~Akka}{\scriptsize\par}

\textcolor{blue}{\scriptsize{}val~a:~ActorRef~=~...~receive(x)~$\Rightarrow$...}{\scriptsize\par}

\textcolor{blue}{\scriptsize{}val~b:~ActorRef~=~...~receive(y)~$\Rightarrow$...}{\scriptsize\par}

\textcolor{blue}{\scriptsize{}a~!~100}{\scriptsize\par}

\textcolor{blue}{\scriptsize{}b~!~1;~~~b~!~2;~~~b~!~3}{\scriptsize\par}
\end{lyxcode}
%
\end{minipage}\hfill{}%
\begin{minipage}[c][1\totalheight][t]{0.45\columnwidth}%
\begin{lyxcode}
\textcolor{blue}{\scriptsize{}//~Chymyst}{\scriptsize\par}

\textcolor{blue}{\scriptsize{}...~go~\{~case~a(x)~$\Rightarrow$~...~\}~}{\scriptsize\par}

\textcolor{blue}{\scriptsize{}...~go~\{~case~b(y)~+~c(z)~$\Rightarrow$~...~\}}{\scriptsize\par}

\textcolor{blue}{\scriptsize{}a(100)}{\scriptsize\par}

\textcolor{blue}{\scriptsize{}b(1);~~b(2);~~b(3);~c(\textquotedbl hello\textquotedbl );}{\scriptsize\par}
\end{lyxcode}
%
\end{minipage}\hfill{}
\begin{itemize}
\item All data resides on messages in mailboxes, is consumed automatically
\item Mailboxes and computations are \emph{values}, can be sent on messages
\item Code for computations can be purely functional
\end{itemize}
\end{frame}

\begin{frame}{The Chemical Metaphor}


\framesubtitle{From real to abstract chemistry}

Real chemistry:
\[
\text{HCl}+\text{NaOH}\rightarrow\text{NaCl}+\text{H}_{2}\text{O}
\]

Abstract chemistry:
\begin{itemize}
\item Chemical ``soup'' contains instances of abstract ``molecules''
\item Combine certain sorts of molecules to start a ``reaction'':
\end{itemize}
~\\

\fbox{\begin{minipage}[c][1\totalheight][t]{0.5\columnwidth}%
\begin{center}
Abstract chemical laws:\\
\texttt{\textcolor{blue}{\footnotesize{}a + b ${\color{blue}\rightarrow}$
a}}\\
\texttt{\textcolor{blue}{\footnotesize{}a + c ${\color{blue}\rightarrow}$
$\textrm{�}$}}
\par\end{center}%
\end{minipage}}\hfill{}%
\begin{minipage}[c][1\totalheight][t]{0.3\columnwidth}%
\includegraphics[width=1\columnwidth]{cham1a}%
\end{minipage}\hfill{}

~\\

\begin{itemize}
\item Program code defines molecules \texttt{\textcolor{blue}{\scriptsize{}a}},
\texttt{\textcolor{blue}{\scriptsize{}b}}, \texttt{\textcolor{blue}{\scriptsize{}c}},
... and chemical laws
\item At initial time, the code emits some molecules into the ``soup''
\item The runtime system evolves the soup \emph{concurrently} and \emph{in
parallel}
\end{itemize}
\end{frame}

\begin{frame}{Chemical Machine in a Nutshell}


\framesubtitle{``Better concurrency through chemistry''}

Translating the chemical metaphor into practice:~\\
~

\fbox{\begin{minipage}[c][1\totalheight][t]{0.5\columnwidth}%
\begin{itemize}
\item Each molecule carries a \textbf{value} (``concurrent data'')
\item Each reaction computes a ``molecule-set-valued'' expression from
input values
\item Resulting molecules are emitted back into the soup
\item Whenever input molecules are available, reactions start concurrently
and in parallel
\end{itemize}
%
\end{minipage}}\hfill{}%
\begin{minipage}[c][1\totalheight][t]{0.45\columnwidth}%
\includegraphics[width=1\columnwidth]{cham2}

~\\
\texttt{\textcolor{blue}{\scriptsize{}site(}}{\scriptsize\par}

\texttt{\textcolor{blue}{\scriptsize{}~ go \{ case }}\texttt{\textbf{\textcolor{blue}{\scriptsize{}a}}}\texttt{\textcolor{blue}{\scriptsize{}(x)
+ }}\texttt{\textbf{\textcolor{blue}{\scriptsize{}b}}}\texttt{\textcolor{blue}{\scriptsize{}(y)
$\Rightarrow$}}{\scriptsize\par}

\texttt{\textcolor{blue}{\scriptsize{}~ ~val z = f(x, y); }}\texttt{\textbf{\textcolor{blue}{\scriptsize{}a}}}\texttt{\textcolor{blue}{\scriptsize{}(z)
\},}}{\scriptsize\par}

\texttt{\textcolor{blue}{\scriptsize{}~ go \{ case }}\texttt{\textbf{\textcolor{blue}{\scriptsize{}a}}}\texttt{\textcolor{blue}{\scriptsize{}(x)
+ }}\texttt{\textbf{\textcolor{blue}{\scriptsize{}c}}}\texttt{\textcolor{blue}{\scriptsize{}(\_)
$\Rightarrow$}}{\scriptsize\par}

\texttt{\textcolor{blue}{\scriptsize{}~ ~ ~ println(x) \}}}{\scriptsize\par}

\texttt{\textcolor{blue}{\scriptsize{})}}{\scriptsize\par}%
\end{minipage}\hfill{}\\
~\\
When a reaction starts: input molecules disappear, expression is computed,
output molecules are emitted
\end{frame}

\begin{frame}{Chemical Machine vs.~Actor model}

\begin{itemize}
\item reaction $\approx$ template for an (auto-started) actor
\item emitted molecule with value $\approx$ message with value, in a mailbox
\item molecule emitters $\approx$ mailbox references
\end{itemize}
Programming with actors: 
\begin{itemize}
\item user code creates and manages explicit actor instances
\item actors typically hold mutable state and/or mutate ``behavior''
\begin{itemize}
\item reasoning is about running processes \emph{and} the data sent on messages
\end{itemize}
\end{itemize}
Programming with the Chemical Machine:
\begin{itemize}
\item processes auto-start when the needed input molecules are available
\item many reactions may start at once, automatically parallel
\begin{itemize}
\item user code does not manipulate references to processes
\begin{itemize}
\item no state, no supervision, no lifecycle to manage
\end{itemize}
\item reasoning is \emph{only} about the \emph{data} currently available
on molecules
\end{itemize}
\end{itemize}
\texttt{Chymyst} code is typically 2x -- 3x shorter than equivalent
Akka code
\end{frame}

\begin{frame}{Example: throttling}

Throttle emitting a molecule \texttt{\textcolor{blue}{\footnotesize{}s(x)}}
with min.~delay of \texttt{\textcolor{blue}{\footnotesize{}delta\_t}}
ms
\begin{lyxcode}
\textcolor{blue}{\footnotesize{}val~r~=~m{[}X{]}}{\footnotesize\par}

\textcolor{blue}{\footnotesize{}val~allow~=~m{[}Unit{]}}{\footnotesize\par}

\textcolor{blue}{\footnotesize{}site(}{\footnotesize\par}

\textcolor{blue}{\footnotesize{}~~go~\{~case~r(x)~+~allow(\_)~\ensuremath{\Rightarrow}}{\footnotesize\par}

\textcolor{blue}{\footnotesize{}~~~~~~~~s(x)}{\footnotesize\par}

\textcolor{blue}{\footnotesize{}~~~~~~~~Thread.sleep(delta\_t)}{\footnotesize\par}

\textcolor{blue}{\footnotesize{}~~~~~~~~allow()}{\footnotesize\par}

\textcolor{blue}{\footnotesize{}~~~~~\}}{\footnotesize\par}

\textcolor{blue}{\footnotesize{})}{\footnotesize\par}

\textcolor{blue}{\footnotesize{}allow()~//~Beginning~of~time;~we~allow~requests.}{\footnotesize\par}
\end{lyxcode}
\begin{itemize}
\item No threads/semaphores/locks, no mutable state
\item External code may emit \texttt{\textcolor{blue}{\footnotesize{}r(x)}}
at will, but \texttt{\textcolor{blue}{\footnotesize{}s(x)}} is then
throttled
\end{itemize}
\href{ https://github.com/softwaremill/akka-vs-scalaz/tree/master/core/src/main/scala/com/softwaremill/ratelimiter}{Implementations in Akka, in Monix, and ZIO}:
about 50 LOC each
\end{frame}

\begin{frame}{Example: parallel merge-sort}

\texttt{Chymyst} code: \href{https://github.com/Chymyst/jc-talk-2017-examples/blob/master/src/test/scala/io/chymyst/talk_examples/MergeSortSpec.scala}{MergeSortSpec.scala}
\begin{lyxcode}
\textcolor{blue}{\scriptsize{}val~mergesort~=~m{[}(Array{[}T{]},~M{[}Array{[}T{]}{]}){]}}{\scriptsize\par}

\textcolor{blue}{\scriptsize{}site(}{\scriptsize\par}

\textcolor{blue}{\scriptsize{}~~go~\{~case~mergesort((arr,~}\textcolor{brown}{\scriptsize{}sortedResult}\textcolor{blue}{\scriptsize{}))~$\Rightarrow$}{\scriptsize\par}

\textcolor{blue}{\scriptsize{}~~~~if~(arr.length~<=~1)~sortedResult(arr)}{\scriptsize\par}

\textcolor{blue}{\scriptsize{}~~~~~~else~\{}{\scriptsize\par}

\textcolor{blue}{\scriptsize{}~~~~~~~~val~}\textcolor{brown}{\scriptsize{}sorted1}\textcolor{blue}{\scriptsize{}~=~m{[}Array{[}T{]}{]}}{\scriptsize\par}

\textcolor{blue}{\scriptsize{}~~~~~~~~val~}\textcolor{brown}{\scriptsize{}sorted2}\textcolor{blue}{\scriptsize{}~=~m{[}Array{[}T{]}{]}}{\scriptsize\par}

\textcolor{blue}{\scriptsize{}~~~~~~~~}\textcolor{brown}{\scriptsize{}site}\textcolor{blue}{\scriptsize{}(}{\scriptsize\par}

\textcolor{blue}{\scriptsize{}~~~~~~~~~~go~\{~case~}\textcolor{brown}{\scriptsize{}sorted1}\textcolor{blue}{\scriptsize{}(x)~+~}\textcolor{brown}{\scriptsize{}sorted2}\textcolor{blue}{\scriptsize{}(y)~$\Rightarrow$~}\textcolor{brown}{\scriptsize{}sortedResult}\textcolor{blue}{\scriptsize{}(arrayMerge(x,y))~\}}{\scriptsize\par}

\textcolor{blue}{\scriptsize{}~~~~~~~~)}{\scriptsize\par}

\textcolor{blue}{\scriptsize{}~~~~~~~~val~(part1,~part2)~=~arr.splitAt(arr.length/2)}{\scriptsize\par}

\textcolor{blue}{\scriptsize{}~~~~~~~~}\textsf{\textcolor{gray}{\footnotesize{}//~Emit~lower-level~}}\textcolor{gray}{\footnotesize{}mergesort}\textsf{\textcolor{gray}{\footnotesize{}~molecules:}}{\footnotesize\par}

\textcolor{blue}{\scriptsize{}~~~~~~~~mergesort(part1,~}\textcolor{brown}{\scriptsize{}sorted1}\textcolor{blue}{\scriptsize{})~+~mergesort(part2,~}\textcolor{brown}{\scriptsize{}sorted2}\textcolor{blue}{\scriptsize{})}{\scriptsize\par}

\textcolor{blue}{\scriptsize{}~~~~\}}{\scriptsize\par}
\end{lyxcode}
\textcolor{blue}{\scriptsize{}\})}\\
~\\
\href{https://gist.github.com/stephenmcd/7edbcfb632c373eaf466}{Implementation in Akka}:
25 LOC for the same functionality
\end{frame}

\begin{frame}{Distributed Chemical Machine}


\framesubtitle{Run code on a cluster with almost no changes}

Distributed Chemical Machine:
\begin{itemize}
\item Same as CM except some molecules are declared as ``distributed''
\item No other code changes necessary!
\end{itemize}
\end{frame}

\begin{frame}{Programs = chemical laws + initial molecules}


\framesubtitle{First example: concurrent counter}

We would like to decrement and increment concurrently

Chemical laws:
\begin{itemize}
\item \textcolor{blue}{\scriptsize{}counter(n) + decr() }\texttt{\textcolor{blue}{\scriptsize{}$\Rightarrow$}}\textcolor{blue}{\scriptsize{}
counter(n -- 1)}{\scriptsize\par}
\item \textcolor{blue}{\scriptsize{}counter(n) + incr() }\texttt{\textcolor{blue}{\scriptsize{}$\Rightarrow$}}\textcolor{blue}{\scriptsize{}
counter(n + 1)}{\scriptsize\par}
\end{itemize}
Initial molecule instances:
\begin{itemize}
\item \textcolor{blue}{\scriptsize{}counter(0)}{\scriptsize\par}
\end{itemize}
``Data stays on the molecules''

We may emit \textcolor{blue}{\scriptsize{}decr()} and \textcolor{blue}{\scriptsize{}incr()}
concurrently
\end{frame}

\begin{frame}{\texttt{Chymyst}: basic features}


\framesubtitle{Molecule emitters and reaction definitions in the Scala DSL}

Define \textbf{molecule} \textbf{emitters}:\\
\texttt{\textcolor{blue}{\scriptsize{}val counter = m{[}Int{]}}}\\
\texttt{\textcolor{blue}{\scriptsize{}val decr = m{[}Unit{]}}}\\
\texttt{\textcolor{blue}{\scriptsize{}val incr = m{[}Unit{]}}}{\scriptsize\par}

~\\

Declare some \textbf{reactions} by pattern match on the molecule values:\\
\texttt{\textcolor{blue}{\scriptsize{}val r0 = go \{ case counter(n)
+ decr(\_) $\Rightarrow$ counter(n-1) \}}} \\
\texttt{\textcolor{blue}{\scriptsize{}val r1 = go \{ case counter(n)
+ incr(\_) $\Rightarrow$ counter(n+1) \}}} 

~\\

Activate a \textbf{reaction site}:\texttt{\textcolor{blue}{\scriptsize{} site(r0,
r1)}}{\scriptsize\par}
\begin{itemize}
\item For brevity, we can define reactions inline, within the \texttt{\textcolor{blue}{\scriptsize{}site()}}
call
\end{itemize}
\end{frame}

\begin{frame}{\texttt{Chymyst}: basic usage}


\framesubtitle{Operational semantics}

Emit some molecules:\\
\texttt{\textcolor{blue}{\scriptsize{}counter(10)}} \textcolor{gray}{\footnotesize{}//
non-blocking side-effect}\\
\texttt{\textcolor{blue}{\scriptsize{}incr()}} \textcolor{gray}{\footnotesize{}//
ditto; we will have}\texttt{\textcolor{gray}{\footnotesize{} counter(11)}}\textcolor{gray}{\footnotesize{}
}\textcolor{gray}{\emph{\footnotesize{}later}} \\
\texttt{\textcolor{blue}{\scriptsize{}incr()}} \textcolor{gray}{\footnotesize{}//
we will have }\texttt{\textcolor{gray}{\footnotesize{}counter(12)}}\textcolor{gray}{\footnotesize{}
}\textcolor{gray}{\emph{\footnotesize{}later}}{\footnotesize\par}
\begin{itemize}
\item Calling \texttt{\textcolor{blue}{\scriptsize{}counter(10)}} returns
\texttt{\textcolor{blue}{\scriptsize{}Unit}} and emits a molecule
as a side-effect
\item This could be the state of the chemical soup at some point in time:
\begin{itemize}
\item \textcolor{blue}{\scriptsize{}counter(10) + incr() + incr()}{\scriptsize\par}
\end{itemize}
\end{itemize}
\end{frame}

\begin{frame}{Concurrent data and concurrent functions}


\framesubtitle{Chemical metaphor vs.~concurrent terms metaphor}
\begin{itemize}
\item Reaction consumes molecules $\approx$ function consumes input values
\item Reaction emits molecules $\approx$ function returns result values
\item Emit molecule with value $\approx$ lift data into the ``concurrent
world''
\item Define reaction $\approx$ lift a function into the ``concurrent
world''
\item Reaction site $\approx$ container for concurrent functions and data
items
\end{itemize}
\end{frame}

\begin{frame}{\texttt{Chymyst}: more features}


\framesubtitle{Blocking vs.~non-blocking molecules}

\textbf{Non-blocking} molecules:
\begin{itemize}
\item emitter \emph{does not wait} until a reaction starts with the new
molecule
\end{itemize}
\textbf{Blocking} molecules:
\begin{itemize}
\item emitter will block until a reaction starts and emits a ``reply value''
\item molecule implicitly carries a pseudo-emitter for ``reply''
\item when the ``reply'' is emitted, its value will be returned to caller
\item Example:\\
~\\
\texttt{\textcolor{blue}{\scriptsize{}val }}\texttt{\textcolor{brown}{\scriptsize{}f}}\texttt{\textcolor{blue}{\scriptsize{}
= b{[}Int, String{]}}}\textcolor{gray}{\footnotesize{} // create blocking
emitter}\texttt{\textcolor{blue}{\scriptsize{}}}~\\
\texttt{\textcolor{blue}{\scriptsize{}go \{ }}\texttt{\textcolor{brown}{\scriptsize{}case
f}}\texttt{\textcolor{blue}{\scriptsize{}(x, }}\texttt{\textcolor{olive}{\scriptsize{}reply}}\texttt{\textcolor{blue}{\scriptsize{})
+ c(y) $\Rightarrow$ }}\texttt{\textcolor{olive}{\scriptsize{}reply}}\texttt{\textcolor{blue}{\scriptsize{}(s\textquotedbl\$\{x
+ y\}\textquotedbl ) \}}}~\\
\texttt{\textcolor{blue}{\scriptsize{}c(100)}}\textcolor{gray}{\footnotesize{}
// non-blocking}\texttt{\textcolor{blue}{\scriptsize{}}}~\\
\texttt{\textcolor{blue}{\scriptsize{}val result: String = }}\texttt{\textcolor{brown}{\scriptsize{}f}}\texttt{\textcolor{blue}{\scriptsize{}(200)}}\textcolor{gray}{\footnotesize{}
// blocking call, will get ``300''}{\footnotesize\par}
\end{itemize}
\end{frame}

\begin{frame}{\texttt{Chymyst}: examples I}


\framesubtitle{Counter with blocking access}

Blocking molecule \texttt{\textcolor{blue}{\scriptsize{}getN}} reads
the value \texttt{\textcolor{blue}{\scriptsize{}x}} in \texttt{\textcolor{blue}{\scriptsize{}counter(x)}}:

~\\
\texttt{\textcolor{blue}{\scriptsize{}val getN = b{[}Unit, Int{]}}}\\
\textcolor{gray}{\footnotesize{}// revise the join definition, appending
this reaction:}\\
\texttt{\textcolor{blue}{\scriptsize{}... val r2 = go \{ case counter(x)
+ getN(\_, reply) $\Rightarrow$ reply(x) \}}}\\
\texttt{\textcolor{blue}{\scriptsize{}site(r0, r1, r2)}}\\
\textcolor{gray}{\footnotesize{}// Emit non-blocking molecules as
before... }\\
\textcolor{gray}{\footnotesize{}// Now emit the blocking molecule:}\\
\texttt{\textcolor{blue}{\scriptsize{}val x = getN()}}\textcolor{gray}{\footnotesize{}
// blocking fetch, returns }\texttt{\textcolor{gray}{\footnotesize{}Int}}\\
~\\
Source code: \href{https://github.com/Chymyst/jc-talk-2017-examples/blob/master/src/test/scala/io/chymyst/talk_examples/CounterSpec.scala}{CounterSpec.scala}
\end{frame}

\begin{frame}{Definitions in local scopes}


\framesubtitle{\texttt{Chymyst} = functional programming + join calculus}

New molecules, reactions, and sites can be defined in \emph{local
scopes}

Emitters (\texttt{\textcolor{blue}{\scriptsize{}read:~M{[}Int{]}}})
can be molecule values too!
\begin{lyxcode}
\textcolor{blue}{\scriptsize{}def~makeCounter(init:~Int):~(M{[}Unit{]},~M{[}M{[}Int{]}{]})~=~\{}{\scriptsize\par}

\textcolor{blue}{\scriptsize{}~~val~c~=~m{[}Int{]}}{\scriptsize\par}

\textcolor{blue}{\scriptsize{}~~val~decr~=~m{[}Unit{]}}{\scriptsize\par}

\textcolor{blue}{\scriptsize{}~~val~get~=~m{[}M{[}Int{]}{]}}{\scriptsize\par}

\textcolor{blue}{\scriptsize{}~~site(~~~~~}{\scriptsize\par}

\textcolor{blue}{\scriptsize{}~~~go~\{~case~c(x)~+~get(read)~\ensuremath{\Rightarrow}~c(x);~read(x)~\},}{\scriptsize\par}

\textcolor{blue}{\scriptsize{}~~~go~\{~case~c(x)~+~decr(\_)~\ensuremath{\Rightarrow}~c(x~-~1)~\}~~}{\scriptsize\par}

\textcolor{blue}{\scriptsize{}~~)}{\scriptsize\par}

\textcolor{blue}{\scriptsize{}~~c(init)}\textcolor{gray}{\scriptsize{}~//~emit~initial~molecule}{\scriptsize\par}

\textcolor{blue}{\scriptsize{}~~(decr,~get)~}\textcolor{gray}{\scriptsize{}//~return~emitters~to~the~outside~scope}{\scriptsize\par}

\textcolor{blue}{\scriptsize{}\}}{\scriptsize\par}

\textcolor{gray}{\scriptsize{}//~usage:}{\scriptsize\par}

\textcolor{blue}{\scriptsize{}val~(decr,~get)~=~makeCounter(100)}{\scriptsize\par}

\textcolor{blue}{\scriptsize{}val~result~=~m{[}Int{]}}{\scriptsize\par}

\textcolor{blue}{\scriptsize{}get(result)}\textcolor{gray}{\scriptsize{}~//~non-blocking~fetch}{\scriptsize\par}
\end{lyxcode}
\end{frame}

\begin{frame}{\texttt{Chymyst}: examples II}


\framesubtitle{Options, Futures, and Map/Reduce}

Implement \texttt{\textcolor{blue}{\scriptsize{}Future}} with blocking
``\texttt{\textcolor{blue}{\scriptsize{}get}}'':\\
\texttt{\textcolor{blue}{\scriptsize{}go \{ case get(\_, reply) $\Rightarrow$
val x = f(); reply(x) \}}}\\
~

Implement Map/Reduce:\\
\texttt{\textcolor{blue}{\scriptsize{}go \{ case c(x) $\Rightarrow$
d(x {*} 2) \}}}\textcolor{gray}{\scriptsize{} // ``map''}\texttt{\textcolor{blue}{\scriptsize{}
}}\\
\texttt{\textcolor{blue}{\scriptsize{}go \{ case res(list) + d(s)
$\Rightarrow$ res(s ::~list) \} }}\textcolor{gray}{\scriptsize{}//
``reduce''}\\
\texttt{\textcolor{blue}{\scriptsize{}go \{ case get(\_, reply) +
res(list) $\Rightarrow$ reply(list) \}}}\\
\texttt{\textcolor{blue}{\scriptsize{}res(Nil)}} \\
\texttt{\textcolor{blue}{\scriptsize{}Seq(1,2,3).foreach(x $\Rightarrow$
c(x))}}\\
\texttt{\textcolor{blue}{\scriptsize{}get()}}\textcolor{gray}{\footnotesize{}
// this returned Seq(4,6,2) in one test}\\
~\\
Source code: \href{https://github.com/Chymyst/jc-talk-2017-examples/blob/master/src/test/scala/io/chymyst/talk_examples/FutureSpec.scala}{FutureSpec.scala}

For more examples, see the \href{https://github.com/Chymyst/chymyst-core}{main repository}
(first-of, barriers, rendezvous, critical sections, readers/writers,
Game of Life, 8 queens, etc.)
\end{frame}

\begin{frame}{\texttt{Chymyst}: examples III}


\framesubtitle{Five Dining Philosophers}

Philosophers \texttt{\textcolor{blue}{\scriptsize{}1, 2, 3, 4, }}\textcolor{blue}{\scriptsize{}5}
and forks \texttt{\textcolor{blue}{\scriptsize{}f12, f23, f34, f45,
f51}}{\scriptsize\par}
\begin{lyxcode}
\textsf{\textcolor{gray}{\footnotesize{}//~...~definitions~of~emitters,~think(),~eat()~omitted~for~brevity}}{\footnotesize\par}

\textcolor{blue}{\scriptsize{}site~(}{\scriptsize\par}

\textcolor{blue}{\scriptsize{}~~go~\{~case~t1(\_)~$\Rightarrow$~think(1);~h1()~\},}{\scriptsize\par}

\textcolor{blue}{\scriptsize{}~~go~\{~case~t2(\_)~$\Rightarrow$~think(2);~h2()~\},}{\scriptsize\par}

\textcolor{blue}{\scriptsize{}~~go~\{~case~t3(\_)~$\Rightarrow$~think(3);~h3()~\},}{\scriptsize\par}

\textcolor{blue}{\scriptsize{}~~go~\{~case~t4(\_)~$\Rightarrow$~think(4);~h4()~\},}{\scriptsize\par}

\textcolor{blue}{\scriptsize{}~~go~\{~case~t5(\_)~$\Rightarrow$~think(5);~h5()~\},}{\scriptsize\par}

~ 

\textcolor{blue}{\scriptsize{}~~go~\{~case~h1(\_)~+~f12(\_)~+~f51(\_)~$\Rightarrow$~eat(1);~t1()~+~f12()~+~f51()~\},}{\scriptsize\par}

\textcolor{blue}{\scriptsize{}~~go~\{~case~h2(\_)~+~f23(\_)~+~f12(\_)~$\Rightarrow$~eat(2);~t2()~+~f23()~+~f12()~\},}{\scriptsize\par}

\textcolor{blue}{\scriptsize{}~~go~\{~case~h3(\_)~+~f34(\_)~+~f23(\_)~$\Rightarrow$~eat(3);~t3()~+~f34()~+~f23()~\},}{\scriptsize\par}

\textcolor{blue}{\scriptsize{}~~go~\{~case~h4(\_)~+~f45(\_)~+~f34(\_)~$\Rightarrow$~eat(4);~t4()~+~f45()~+~f34()~\},}{\scriptsize\par}

\textcolor{blue}{\scriptsize{}~~go~\{~case~h5(\_)~+~f51(\_)~+~f45(\_)~$\Rightarrow$~eat(5);~t5()~+~f51()~+~f45()~\}}{\scriptsize\par}

\textcolor{blue}{\scriptsize{})}{\scriptsize\par}

\textcolor{blue}{\scriptsize{}t1()~+~t2()~+~t3()~+~t4()~+~t5()}{\scriptsize\par}

\textcolor{blue}{\scriptsize{}f12()~+~f23()~+~f34()~+~f45()~+~f51()}~\\
\end{lyxcode}
Source code: \href{https://github.com/Chymyst/jc-talk-2017-examples/blob/master/src/main/scala/io/chymyst/talk_examples/DiningPhilosophers.scala}{DiningPhilosophers.scala}
\end{frame}

\begin{frame}{\texttt{Chymyst}: examples IV}


\framesubtitle{Concurrent merge-sort: chemistry pseudocode}

The \texttt{\textcolor{blue}{\scriptsize{}mergesort}} molecule starts
a ``chain reaction'':
\begin{itemize}
\item receives the upper-level ``\texttt{\textcolor{brown}{\scriptsize{}sortedResult}}''
molecule
\item defines its own ``\texttt{\textcolor{brown}{\scriptsize{}sorted}}''
molecules in \emph{local scope}
\item emits upper-level ``\texttt{\textcolor{brown}{\scriptsize{}sortedResult}}''
when done
\end{itemize}
\begin{lyxcode}
\textcolor{blue}{\scriptsize{}mergesort(~(arr,~}\textcolor{brown}{\scriptsize{}sortedResult}\textcolor{blue}{\scriptsize{})~)~$\Rightarrow$}{\scriptsize\par}

\textcolor{blue}{\scriptsize{}~~~~~~~val~(part1,~part2)~=~arr.splitAt(arr.length/2)}{\scriptsize\par}

\textcolor{blue}{\scriptsize{}~~~~~~~}\textcolor{brown}{\scriptsize{}sorted1}\textcolor{blue}{\scriptsize{}(x)~+~}\textcolor{brown}{\scriptsize{}sorted2}\textcolor{blue}{\scriptsize{}(y)~$\Rightarrow$~}\textcolor{brown}{\scriptsize{}sortedResult}\textcolor{blue}{\scriptsize{}(~arrayMerge(x,y)~)}{\scriptsize\par}

~~~

\textsf{\textcolor{gray}{\footnotesize{}~~~~~~~//~Emit~lower-level~}}\textcolor{gray}{\footnotesize{}mergesort}\textsf{\textcolor{gray}{\footnotesize{}~molecules:}}{\footnotesize\par}

\textcolor{blue}{\scriptsize{}~~~~~~~mergesort(part1,~}\textcolor{brown}{\scriptsize{}sorted1}\textcolor{blue}{\scriptsize{})~+~mergesort(part2,~}\textcolor{brown}{\scriptsize{}sorted2}\textcolor{blue}{\scriptsize{})}{\scriptsize\par}

\end{lyxcode}
\end{frame}

\begin{frame}{Pipelined molecules: An automatic optimization}


\framesubtitle{In join calculus: channels with \emph{ordered} mailboxes }

Reaction scheduler in \texttt{Chymyst}:
\begin{itemize}
\item Examines all present molecule instances and runs the next reaction 
\begin{itemize}
\item Is it sufficient to examine \emph{only one} molecule instance?
\begin{itemize}
\item This could be true or false depending on the specific molecule
\end{itemize}
\end{itemize}
\item If true, the molecule's instances are held in a queue (``pipelined'')
\begin{itemize}
\item \texttt{Chymyst} automatically makes \emph{some} molecules pipelined
\end{itemize}
\end{itemize}
In a given chemical program, can we pipeline the molecule \texttt{\textcolor{blue}{\footnotesize{}a(x)}}?
\begin{itemize}
\item Consider the predicate \texttt{\textcolor{blue}{\footnotesize{}f(x,
y, z,...)}} for starting a reaction, e.g.:\\
\texttt{\textcolor{blue}{\footnotesize{} f(x, y, b) = HAVE(b(y)) \&\&
x == 0 \&\& y > x}}{\footnotesize\par}
\item The predicate \texttt{\textcolor{blue}{\footnotesize{}f(...)}} must
be \emph{factorizable} into a conjuction:\\
\texttt{\textcolor{blue}{\footnotesize{} f(x, y, z, ...) = p(x) \&\&
q(y, z,...)}}{\footnotesize\par}
\begin{itemize}
\item I have a proof that this optimization preserves semantics
\end{itemize}
\end{itemize}
\end{frame}

\begin{frame}{Everything you need to know about join calculus...}


\framesubtitle{... but the \href{https://en.wikipedia.org/wiki/Join-calculus}{Wikipedia page}
confused you, so you were afraid to ask}

Academic descriptions of JC use the ``message/channel'' terminology

\texttt{\footnotesize{}\bigskip{}
}{\footnotesize\par}
\begin{center}
\begin{tabular}{|c|c|c|}
\hline 
\textbf{Chymyst} & \textbf{join calculus} & \textbf{code}\tabularnewline
\hline 
\hline 
molecule & message on channel & \texttt{\textcolor{blue}{\footnotesize{}a(123)}}\textcolor{gray}{\footnotesize{}
// side effect}\tabularnewline
\hline 
emitter & channel name & \texttt{\textcolor{blue}{\footnotesize{}val a:~M{[}Int{]}}}\tabularnewline
\hline 
blocking emitter & blocking channel & \texttt{\textcolor{blue}{\footnotesize{}val q:~B{[}Unit, Int{]}}}\tabularnewline
\hline 
reaction & process & \texttt{\textcolor{blue}{\footnotesize{}go \{ case a(x) + ... \}}}\tabularnewline
\hline 
emitting a molecule & sending a message & \texttt{\textcolor{blue}{\footnotesize{}a(123)}}\textcolor{gray}{\footnotesize{}
// side effect}\tabularnewline
\hline 
reaction site & join definition & \texttt{\textcolor{blue}{\footnotesize{}site(r1, r2, ...)}}\tabularnewline
\hline 
\end{tabular}
\par\end{center}

\end{frame}

\begin{frame}{Join Calculus in the wild}

\begin{itemize}
\item Previous implementations:
\begin{itemize}
\item Funnel {[}\href{http://lampwww.epfl.ch/funnel/}{M. Odersky et al., 2000}{]}
\item Join Java {[}\href{http://www.vonitzstein.com/Project_JoinJava.html}{von Itzstein et al., 2001-2005}{]}
\item JOCaml  (\href{http://jocaml.inria.fr}{jocaml.inria.fr}) {[}\href{http://research.microsoft.com/en-us/um/people/fournet/papers/jocaml-afp4-summer-school-02.pdf}{Fournet et al.�2003}{]}
\item ``Join in Scala'' compiler patch {[}\href{http://lampwww.epfl.ch/~cremet/misc/join_in_scala/index.html}{V. Cremet 2003}{]}
\item Joins library for .NET {[}\href{http://research.microsoft.com/en-us/um/people/crusso/joins/}{P. Crusso 2006}{]}
\item ScalaJoins {[}\href{http://lampwww.epfl.ch/~phaller/joins/index.html}{P. Haller 2008}{]}
\item Joinads (F\#, Haskell) {[}\href{https://www.microsoft.com/en-us/research/publication/joinads-a-retargetable-control-flow-construct-for-reactive-parallel-and-concurrent-programming/}{Petricek and Syme 2011}{]}
\item ScalaJoin {[}\href{https://github.com/Jiansen/ScalaJoin}{J. He 2011}{]}
\item \href{https://github.com/winitzki/CocoaJoin}{CocoaJoin (iOS)}, \href{https://github.com/winitzki/AndroJoin}{AndroJoin (Android)}
{[}S.W.\texttt{\textcolor{blue}{\scriptsize{}~}}2013{]}
\item \href{http://guidosalva.github.io/REScala/jescala/}{JEScala} {[}G.
Salvaneschi 2014{]}
\end{itemize}
\item \href{https://github.com/chymyst/chymyst-core}{Chymyst} -{}- a new
JC implementation in Scala (this talk)
\begin{itemize}
\item Better syntax, more checks of code sanity
\item (Some) automatic fault tolerance
\item Thread pool and thread priority control
\item Event monitoring and unit testing APIs
\end{itemize}
\end{itemize}
\end{frame}

\begin{frame}{Conclusions and outlook}

\begin{itemize}
\item Chemical machine = declarative, purely functional concurrency
\begin{itemize}
\item Similar to ``Actors'', but easier to use and ``more purely functional''
\item Short, declarative code implementing barriers, rendezvous, etc.
\end{itemize}
\item A new open-source Scala implementation: \href{https://github.com/Chymyst/chymyst-core}{Chymyst}
\begin{itemize}
\item Full-featured implementation of join calculus
\item Static DSL code analysis (with Scala macros)
\item Industry-strength features (thread priority control, pipelining, fault
tolerance, unit testing and debugging APIs)
\item Extensive documentation: \href{https://winitzki.gitbooks.io/concurrency-in-reactions-declarative-multicore-in/content/}{tutorial book}
and \href{https://github.com/winitzki/talks/blob/master/join-calculus-paper/join-calculus-paper.pdf}{draft paper}
\end{itemize}
\item On the future roadmap:
\begin{itemize}
\item Thread fusion for better performance
\item Full continuation-passing transformation to nonblocking code
\item Automatic backpressure (``reaction temperature'')
\item Automatic distributed runtime (``distributed soup'')
\end{itemize}
\item Example code for \href{https://github.com/winitzki/talks/blob/master/join_calculus/join_calculus_2017_SBTB.pdf}{this talk}:
{\footnotesize{}\href{https://github.com/Chymyst/jc-talk-2017-examples}{github.com/Chymyst/jc-talk-2017-examples}}{\footnotesize\par}
\end{itemize}
\end{frame}

\end{document}
}{github.com/Chymyst/jc-talk-2017-examples}}{\footnotesize\par}
\end{itemize}
\end{frame}

\end{document}
