\batchmode
\makeatletter
\def\input@path{{/Users/sergei.winitzki/Code/talks/join_calculus/}}
\makeatother
\documentclass[english]{beamer}
\usepackage[T1]{fontenc}
\usepackage[latin9]{inputenc}
\setcounter{secnumdepth}{3}
\setcounter{tocdepth}{3}
\usepackage{babel}
\usepackage{array}
\usepackage{calc}
\usepackage{amstext}
\usepackage{graphicx}
\ifx\hypersetup\undefined
  \AtBeginDocument{%
    \hypersetup{unicode=true,pdfusetitle,
 bookmarks=true,bookmarksnumbered=false,bookmarksopen=false,
 breaklinks=false,pdfborder={0 0 1},backref=false,colorlinks=true}
  }
\else
  \hypersetup{unicode=true,pdfusetitle,
 bookmarks=true,bookmarksnumbered=false,bookmarksopen=false,
 breaklinks=false,pdfborder={0 0 1},backref=false,colorlinks=true}
\fi

\makeatletter

%%%%%%%%%%%%%%%%%%%%%%%%%%%%%% LyX specific LaTeX commands.
%% Because html converters don't know tabularnewline
\providecommand{\tabularnewline}{\\}

%%%%%%%%%%%%%%%%%%%%%%%%%%%%%% Textclass specific LaTeX commands.
% this default might be overridden by plain title style
\newcommand\makebeamertitle{\frame{\maketitle}}%
% (ERT) argument for the TOC
\AtBeginDocument{%
  \let\origtableofcontents=\tableofcontents
  \def\tableofcontents{\@ifnextchar[{\origtableofcontents}{\gobbletableofcontents}}
  \def\gobbletableofcontents#1{\origtableofcontents}
}
\newenvironment{lyxcode}
  {\par\begin{list}{}{
    \setlength{\rightmargin}{\leftmargin}
    \setlength{\listparindent}{0pt}% needed for AMS classes
    \raggedright
    \setlength{\itemsep}{0pt}
    \setlength{\parsep}{0pt}
    \normalfont\ttfamily}%
   \def\{{\char`\{}
   \def\}{\char`\}}
   \def\textasciitilde{\char`\~}
   \item[]}
  {\end{list}}

%%%%%%%%%%%%%%%%%%%%%%%%%%%%%% User specified LaTeX commands.
\usetheme[secheader]{Boadilla}
\usecolortheme{seahorse}
\author{Sergei Winitzki}
\date{November 16, 2018}
\setbeamertemplate{headline}{} % disable headline at top
\setbeamertemplate{navigation symbols}{} % disable navigation bar at bottom
\title[Declarative distributed concurrency]{Declarative distributed concurrency in Scala}
\institute[SBTB 2018]{Scale by the Bay 2018}

\makeatother

\begin{document}
\frame{\titlepage}
\begin{frame}{Talk summary}


\framesubtitle{How I learned to forget semaphores and to love concurrency}

\texttt{Chymyst} = an implementation of the Chemical Machine (CM)
paradigm
\begin{itemize}
\item CM $\approx$ Actors made purely functional and auto-parallelized
\item Intuitions about why CM works better than other concurrency models
\begin{itemize}
\item Comparison with related work: ING Baker, BPMN (workflow)
\end{itemize}
\item New extension for distributed programming: DCM
\item Code examples and demos
\end{itemize}
Not in this talk: academic theory
\begin{itemize}
\item Petri nets, $\pi$-calculus, join calculus, joinads, mobile agent
calculus...
\item DCM formulated within some theory of distributed programming
\end{itemize}
\end{frame}

\begin{frame}{Concurrent \& parallel programming: How we cope}

\emph{Imperative} concurrency \& parallelism is difficult to reason
about:
\begin{itemize}
\item low-level API: callbacks, threads, semaphores, mutex locks
\item hard to reason about mutable state and running processes
\item hard to test -- non-deterministic runtime behavior!
\begin{itemize}
\item race conditions, deadlocks, livelocks
\end{itemize}
\end{itemize}
\begin{center}
\vspace{-0.2cm}
Known declarative approaches to avoid these problems:\vspace{-0.5cm}
\par\end{center}

\begin{center}
\begin{tabular}{|>{\centering}p{0.35\textwidth}|c|>{\centering}p{0.3\textwidth}|}
\hline 
\textbf{\footnotesize{}Kind of concurrency} &
\textbf{\footnotesize{}Formal structure} &
\textbf{\footnotesize{}Scala implementation}\tabularnewline
\hline 
\hline 
synchronous parallelism &
applicative functor &
Spark, \texttt{\textcolor{blue}{\scriptsize{}.par.map()}}\tabularnewline
\hline 
asynchronous streaming DAG &
monadic functor &
\texttt{\textcolor{blue}{\scriptsize{}Future}}, \texttt{\textcolor{blue}{\scriptsize{}async}}{\footnotesize{}/}\texttt{\textcolor{blue}{\scriptsize{}await}},
\texttt{\footnotesize{}RxJava}, Akka Streams\tabularnewline
\hline 
unrestricted streaming &
recursive monad+ &
Flink, fs2, ZIO\tabularnewline
\hline 
unrestricted concurrency &
? &
Akka, \texttt{\footnotesize{}Chymyst}\tabularnewline
\hline 
\end{tabular}
\par\end{center}

For distributed computing: challenges remain
\begin{itemize}
\item coordination and consensus, persistence and fault tolerance
\item cluster configuration and discovery
\begin{itemize}
\item distributed coordination as a service: Apache ZooKeeper, \texttt{etcd}
\end{itemize}
\end{itemize}
\end{frame}

\begin{frame}{``Dining philosophers''}


\framesubtitle{The paradigmatic problem of concurrency, parallelism and resource
contention}

\href{https://en.wikipedia.org/wiki/Dining_philosophers_problem}{Five philosophers sit at a round table},
taking turns eating and thinking for random time intervals
\begin{center}
\includegraphics[height=4cm]{3_Users_sergei_win\includegraphics[height=4cm]{An_illustration_of_the_dining_philosophers_problem}
\par\end{center}

Problem: simulate the process, avoiding deadlock and starvation

Solutions in various programming languages: see \href{https://rosettacode.org/wiki/Dining_philosophers}{Rosetta Code}
\begin{itemize}
\item Can this be implemented via (effectful) streams? (I think not.)
\item The Chemical Machine code is purely declarative
\end{itemize}
\end{frame}

\begin{frame}{Chemical Machine vs.~AWS Lambda}

The Chemical Machine paradigm:
\begin{itemize}
\item A \emph{declarative language} for concurrent and parallel computations
\begin{itemize}
\item largely unknown and unused by the software engineering community
\item \texttt{Chymyst} -- an \href{https://github.com/Chymyst/chymyst-core}{open-source library \& embedded DSL}
for Scala
\item presented in my SBTB talks in 2016 and 2017
\end{itemize}
\end{itemize}
AWS Lambda
\begin{itemize}
\item wait for an event, signalling arrival of input data
\item run computation when input data becomes available
\item the computation is automatically parallelized, data-driven
\item writing output will create a new event
\end{itemize}
Modify the AWS Lambda model by adding new requirements:
\begin{itemize}
\item a Lambda should be able to wait for several \emph{unrelated} events
\item several Lambdas contend \emph{atomically} on shared input events
\end{itemize}
With these new requirements, AWS Lambda becomes a purely functional
\emph{unrestricted} \emph{concurrency} model
\end{frame}

\begin{frame}{The Chemical Machine vs.~the Actor model}

Modify the Actor model by adding new requirements:
\begin{itemize}
\item when messages arrive, actors are auto-created, maybe \emph{in parallel}
\item actors may wait atomically for messages in \emph{several} different
mailboxes
\end{itemize}
It follows from these requirements that... 
\begin{itemize}
\item Auto-created actor instances are \emph{stateless} and invisible to
user
\item User code defines \emph{mailboxes} and \emph{computations} that consume
messages
\item Repeated messages may be consumed in parallel
\item Messages are sent to mailboxes, not to specific actor instances:
\end{itemize}
\begin{minipage}[c][1\totalheight][t]{0.5\columnwidth}%
\begin{lyxcode}
\textcolor{blue}{\scriptsize{}//~Akka}{\scriptsize\par}

\textcolor{blue}{\scriptsize{}val~a:~ActorRef~=~...~receive(x)~$\Rightarrow$...}{\scriptsize\par}

\textcolor{blue}{\scriptsize{}val~b:~ActorRef~=~...~receive(y)~$\Rightarrow$...}{\scriptsize\par}

\textcolor{blue}{\scriptsize{}a~!~100}{\scriptsize\par}

\textcolor{blue}{\scriptsize{}b~!~1;~~~b~!~2;~~~b~!~3}{\scriptsize\par}
\end{lyxcode}
%
\end{minipage}\hfill{}%
\begin{minipage}[c][1\totalheight][t]{0.45\columnwidth}%
\begin{lyxcode}
\textcolor{blue}{\scriptsize{}//~Chymyst}{\scriptsize\par}

\textcolor{blue}{\scriptsize{}...~go~\{~case~a(x)~$\Rightarrow$~...~\}~}{\scriptsize\par}

\textcolor{blue}{\scriptsize{}...~go~\{~case~b(y)~+~c(z)~$\Rightarrow$~...~\}}{\scriptsize\par}

\textcolor{blue}{\scriptsize{}a(100)}{\scriptsize\par}

\textcolor{blue}{\scriptsize{}b(1);~~b(2);~~b(3);~c(\textquotedbl hello\textquotedbl );}{\scriptsize\par}
\end{lyxcode}
%
\end{minipage}\hfill{}
\begin{itemize}
\item All data resides on messages in mailboxes, is consumed automatically
\item Mailboxes and computations are \emph{values}, can be sent on messages
\end{itemize}
Any Actor program can be straightforwardly translated into CM
\end{frame}

\begin{frame}{The chemical metaphor}


\framesubtitle{From real to abstract chemistry}

Real chemistry:
\[
\text{HCl}+\text{NaOH}\rightarrow\text{NaCl}+\text{H}_{2}\text{O}
\]

Abstract chemistry:
\begin{itemize}
\item Abstract ``molecules'' float around in a ``chemical reaction site''
\item Certain sorts of molecules may combine to start a ``reaction'':
\end{itemize}
~\\

\fbox{\begin{minipage}[c][1\totalheight][t]{0.5\columnwidth}%
\begin{center}
Abstract chemical laws:\\
\texttt{\textcolor{blue}{\footnotesize{}a + b ${\color{blue}\rightarrow}$
a}}\\
\texttt{\textcolor{blue}{\footnotesize{}a + c ${\color{blue}\rightarrow}$
$\textrm{�}$}}
\par\end{center}%
\end{minipage}}\hfill{}%
\begin{minipage}[c][1\totalheight][t]{0.3\columnwidth}%
\includegraphics[width=1\columnwidth]{cham1a}%
\end{minipage}\hfill{}

~\\

\begin{itemize}
\item Program code defines molecules \texttt{\textcolor{blue}{\scriptsize{}a}},
\texttt{\textcolor{blue}{\scriptsize{}b}}, \texttt{\textcolor{blue}{\scriptsize{}c}},
... and chemical laws
\item At initial time, the code emits some molecules into the site
\item The runtime system evolves the molecules \emph{concurrently} and \emph{in
parallel}
\end{itemize}
\end{frame}

\begin{frame}{Chemical Machine in a nutshell}


\framesubtitle{``Better concurrency through chemistry''}

Translating the chemical metaphor into practice:~\\
~

\fbox{\begin{minipage}[c][1\totalheight][t]{0.5\columnwidth}%
\begin{itemize}
\item Each molecule carries a \textbf{value} (``concurrent data'')
\item Each reaction computes new values from its input values
\item Some molecules with new values may be emitted back into the reaction
site
\end{itemize}
%
\end{minipage}}\hfill{}%
\begin{minipage}[c][1\totalheight][t]{0.45\columnwidth}%
\includegraphics[width=1\columnwidth]{cham2}

~\\
\texttt{\textcolor{blue}{\scriptsize{}site(}}{\scriptsize\par}

\texttt{\textcolor{blue}{\scriptsize{}~ go \{ case }}\texttt{\textbf{\textcolor{blue}{\scriptsize{}a}}}\texttt{\textcolor{blue}{\scriptsize{}(x)
+ }}\texttt{\textbf{\textcolor{blue}{\scriptsize{}b}}}\texttt{\textcolor{blue}{\scriptsize{}(y)
$\Rightarrow$}}{\scriptsize\par}

\texttt{\textcolor{blue}{\scriptsize{}~ ~val z = f(x, y); }}\texttt{\textbf{\textcolor{blue}{\scriptsize{}a}}}\texttt{\textcolor{blue}{\scriptsize{}(z)
\},}}{\scriptsize\par}

\texttt{\textcolor{blue}{\scriptsize{}~ go \{ case }}\texttt{\textbf{\textcolor{blue}{\scriptsize{}a}}}\texttt{\textcolor{blue}{\scriptsize{}(x)
+ }}\texttt{\textbf{\textcolor{blue}{\scriptsize{}c}}}\texttt{\textcolor{blue}{\scriptsize{}(\_)
$\Rightarrow$}}{\scriptsize\par}

\texttt{\textcolor{blue}{\scriptsize{}~ ~ ~ println(x) \}}}{\scriptsize\par}

\texttt{\textcolor{blue}{\scriptsize{})}}{\scriptsize\par}%
\end{minipage}\hfill{}\\
~\\
When a reaction starts: input molecules disappear, new values are
computed, output molecules are emitted

Reactions are functions from input values to output values
\end{frame}

\begin{frame}{Chemical Machine vs.~Actor model}

\begin{itemize}
\item reaction $\approx$ template for an (auto-started) actor
\item emitted molecule with value $\approx$ message with value, in a mailbox
\item molecule emitters $\approx$ mailbox references
\end{itemize}
Programming with actors: 
\begin{itemize}
\item user code creates and manages explicit actor instances
\item actors typically hold mutable state and/or mutate ``behavior''
\begin{itemize}
\item reasoning is about running processes \emph{and} the data sent on messages
\end{itemize}
\end{itemize}
Programming with the Chemical Machine:
\begin{itemize}
\item processes auto-start when the needed input molecules are available
\item many reactions may start at once, automatically parallel
\begin{itemize}
\item user code does not manipulate references to processes
\begin{itemize}
\item no state, no supervision, no lifecycle to manage
\end{itemize}
\item reasoning is \emph{only} about the \emph{data} currently available
on molecules
\end{itemize}
\end{itemize}
\texttt{Chymyst} code is typically 2x -- 3x shorter than equivalent
Akka code
\end{frame}

\begin{frame}{Example: throttling}

Throttle emitting a molecule \texttt{\textcolor{blue}{\footnotesize{}s(x)}}
with min.~delay of \texttt{\textcolor{blue}{\footnotesize{}delta}}
ms
\begin{lyxcode}
\textcolor{blue}{\footnotesize{}def~throttle{[}X{]}(s:~M{[}X{]},~delta:~Long):~M{[}X{]}~=~\{}{\footnotesize\par}

\textcolor{blue}{\footnotesize{}~val~r~=~m{[}X{]}}{\footnotesize\par}

\textcolor{blue}{\footnotesize{}~val~allow~=~m{[}Unit{]}}{\footnotesize\par}

\textcolor{blue}{\footnotesize{}~site(}{\footnotesize\par}

\textcolor{blue}{\footnotesize{}~~go~\{~case~r(x)~+~allow(\_)~\ensuremath{\Rightarrow}}{\footnotesize\par}

\textcolor{blue}{\footnotesize{}~~~~~~~~s(x)}{\footnotesize\par}

\textcolor{blue}{\footnotesize{}~~~~~~~~Thread.sleep(delta)}{\footnotesize\par}

\textcolor{blue}{\footnotesize{}~~~~~~~~allow()}{\footnotesize\par}

\textcolor{blue}{\footnotesize{}~~~~~\}}{\footnotesize\par}

\textcolor{blue}{\footnotesize{}~)}{\footnotesize\par}

\textcolor{blue}{\footnotesize{}~allow()~//~Beginning~of~time;~we~allow~requests.}{\footnotesize\par}

\textcolor{blue}{\footnotesize{}~r}{\footnotesize\par}

\textcolor{blue}{\footnotesize{}\}}{\footnotesize\par}
\end{lyxcode}
\begin{itemize}
\item No threads/semaphores/locks, no mutable state
\item External code may emit \texttt{\textcolor{blue}{\footnotesize{}r(x)}}
at will, and \texttt{\textcolor{blue}{\footnotesize{}s(x)}} is then
throttled
\end{itemize}
\href{ https://github.com/softwaremill/akka-vs-scalaz/tree/master/core/src/main/scala/com/softwaremill/ratelimiter}{Implementations in Akka, in Monix, and ZIO}:
about 50 LOC each
\end{frame}

\begin{frame}{Example: map/reduce}

A simple map/reduce implementation:
\begin{lyxcode}
\textcolor{blue}{\scriptsize{}val~c~=~m{[}A{]}}\textcolor{gray}{\scriptsize{}~//~Initial~values~have~type~`A`.}{\scriptsize\par}

\textcolor{blue}{\scriptsize{}val~d~=~m{[}(Int,~B){]}~}\textcolor{gray}{\scriptsize{}//~`B`~is~a~commutative~monoid.}{\scriptsize\par}

\textcolor{blue}{\scriptsize{}val~res~=~m{[}B{]}}\textcolor{gray}{\scriptsize{}~//~Final~result~of~type~`B`.}{\scriptsize\par}

\textcolor{blue}{\scriptsize{}val~fetch~=~b{[}Unit,~B{]}}\textcolor{gray}{\scriptsize{}~//~Blocking~emitter.}{\scriptsize\par}

\textcolor{blue}{\scriptsize{}site(}{\scriptsize\par}

\textcolor{gray}{\scriptsize{}~~//~``map''}{\scriptsize\par}

\textcolor{blue}{\scriptsize{}~~go~\{~case~c(x)~$\Rightarrow$~d((1,~long\_computation(x)))~\},~}{\scriptsize\par}

\textcolor{blue}{\scriptsize{}~}\textcolor{gray}{\scriptsize{}~//~``reduce''}{\scriptsize\par}

\textcolor{blue}{\scriptsize{}~~go~\{~case~d((n1,~b1))~+~d((n2,~b2))~$\Rightarrow$}{\scriptsize\par}

\textcolor{blue}{\scriptsize{}~~~val~(newN,~newB)~=~(n1~+~n2,~b1~|+|~b2)}{\scriptsize\par}

\textcolor{blue}{\scriptsize{}~~~if~(newN~==~total)~res(newB)~else~d((newN,~newB))~}{\scriptsize\par}

\textcolor{blue}{\scriptsize{}~~\},}{\scriptsize\par}

\textcolor{blue}{\scriptsize{}~~go~\{~case~fetch(\_,~reply)~+~res(b)~$\Rightarrow$~reply(b)~\}}{\scriptsize\par}

\textcolor{blue}{\scriptsize{})}{\scriptsize\par}

\textcolor{blue}{\scriptsize{}(1~to~100).foreach(x~$\Rightarrow$~c(x))}{\scriptsize\par}

\textcolor{blue}{\scriptsize{}fetch()}\textcolor{gray}{\footnotesize{}~//~Blocking~call~returns~the~final~result.}{\footnotesize\par}
\end{lyxcode}
~\\
Compare with the \href{https://stackoverflow.com/questions/17291851/mapreduce-implementation-with-akka}{Akka example}
(100+ LOC)
\end{frame}

\begin{frame}{Example: parallel merge-sort}

\texttt{Chymyst} code: \href{https://github.com/Chymyst/jc-talk-2017-examples/blob/master/src/test/scala/io/chymyst/talk_examples/MergeSortSpec.scala}{MergeSortSpec.scala}
\begin{lyxcode}
\textcolor{blue}{\scriptsize{}val~mergesort~=~m{[}(Array{[}T{]},~M{[}Array{[}T{]}{]}){]}}{\scriptsize\par}

\textcolor{blue}{\scriptsize{}site(}{\scriptsize\par}

\textcolor{blue}{\scriptsize{}~~go~\{~case~mergesort((arr,~}\textcolor{brown}{\scriptsize{}sortedResult}\textcolor{blue}{\scriptsize{}))~$\Rightarrow$}{\scriptsize\par}

\textcolor{blue}{\scriptsize{}~~~~if~(arr.length~<=~1)~sortedResult(arr)}{\scriptsize\par}

\textcolor{blue}{\scriptsize{}~~~~~~else~\{}{\scriptsize\par}

\textcolor{blue}{\scriptsize{}~~~~~~~~val~}\textcolor{brown}{\scriptsize{}sorted1}\textcolor{blue}{\scriptsize{}~=~m{[}Array{[}T{]}{]}}{\scriptsize\par}

\textcolor{blue}{\scriptsize{}~~~~~~~~val~}\textcolor{brown}{\scriptsize{}sorted2}\textcolor{blue}{\scriptsize{}~=~m{[}Array{[}T{]}{]}}{\scriptsize\par}

\textcolor{blue}{\scriptsize{}~~~~~~~~}\textcolor{brown}{\scriptsize{}site}\textcolor{blue}{\scriptsize{}(}{\scriptsize\par}

\textcolor{blue}{\scriptsize{}~~~~~~~~~~go~\{~case~}\textcolor{brown}{\scriptsize{}sorted1}\textcolor{blue}{\scriptsize{}(x)~+~}\textcolor{brown}{\scriptsize{}sorted2}\textcolor{blue}{\scriptsize{}(y)~$\Rightarrow$~}\textcolor{brown}{\scriptsize{}sortedResult}\textcolor{blue}{\scriptsize{}(arrayMerge(x,y))~\}}{\scriptsize\par}

\textcolor{blue}{\scriptsize{}~~~~~~~~)}{\scriptsize\par}

\textcolor{blue}{\scriptsize{}~~~~~~~~val~(part1,~part2)~=~arr.splitAt(arr.length/2)}{\scriptsize\par}

\textcolor{blue}{\scriptsize{}~~~~~~~~}\textsf{\textcolor{gray}{\footnotesize{}//~Emit~lower-level~}}\textcolor{gray}{\footnotesize{}mergesort}\textsf{\textcolor{gray}{\footnotesize{}~molecules:}}{\footnotesize\par}

\textcolor{blue}{\scriptsize{}~~~~~~~~mergesort(part1,~}\textcolor{brown}{\scriptsize{}sorted1}\textcolor{blue}{\scriptsize{})~+~mergesort(part2,~}\textcolor{brown}{\scriptsize{}sorted2}\textcolor{blue}{\scriptsize{})}{\scriptsize\par}

\textcolor{blue}{\scriptsize{}~~~~\}}{\scriptsize\par}

\textcolor{blue}{\scriptsize{}\})}{\scriptsize\par}
\end{lyxcode}
~\\
\href{https://gist.github.com/stephenmcd/7edbcfb632c373eaf466}{Implementation in Akka}:
25 LOC for the same functionality
\end{frame}

\begin{frame}{Example: Dining philosophers}


\framesubtitle{Five Dining Philosophers}

Philosophers \texttt{\textcolor{blue}{\scriptsize{}1, 2, 3, 4, }}\textcolor{blue}{\scriptsize{}5}
and forks \texttt{\textcolor{blue}{\scriptsize{}f12, f23, f34, f45,
f51}}{\scriptsize\par}
\begin{lyxcode}
\textsf{\textcolor{gray}{\footnotesize{}//~...~definitions~of~emitters,~think(),~eat()~omitted~for~brevity}}{\footnotesize\par}

\textcolor{blue}{\scriptsize{}site~(}{\scriptsize\par}

\textcolor{blue}{\scriptsize{}~~go~\{~case~t1(\_)~$\Rightarrow$~think(1);~h1()~\},}{\scriptsize\par}

\textcolor{blue}{\scriptsize{}~~go~\{~case~t2(\_)~$\Rightarrow$~think(2);~h2()~\},}{\scriptsize\par}

\textcolor{blue}{\scriptsize{}~~go~\{~case~t3(\_)~$\Rightarrow$~think(3);~h3()~\},}{\scriptsize\par}

\textcolor{blue}{\scriptsize{}~~go~\{~case~t4(\_)~$\Rightarrow$~think(4);~h4()~\},}{\scriptsize\par}

\textcolor{blue}{\scriptsize{}~~go~\{~case~t5(\_)~$\Rightarrow$~think(5);~h5()~\},}{\scriptsize\par}

~ 

\textcolor{blue}{\scriptsize{}~~go~\{~case~h1(\_)~+~f12(\_)~+~f51(\_)~$\Rightarrow$~eat(1);~t1()~+~f12()~+~f51()~\},}{\scriptsize\par}

\textcolor{blue}{\scriptsize{}~~go~\{~case~h2(\_)~+~f23(\_)~+~f12(\_)~$\Rightarrow$~eat(2);~t2()~+~f23()~+~f12()~\},}{\scriptsize\par}

\textcolor{blue}{\scriptsize{}~~go~\{~case~h3(\_)~+~f34(\_)~+~f23(\_)~$\Rightarrow$~eat(3);~t3()~+~f34()~+~f23()~\},}{\scriptsize\par}

\textcolor{blue}{\scriptsize{}~~go~\{~case~h4(\_)~+~f45(\_)~+~f34(\_)~$\Rightarrow$~eat(4);~t4()~+~f45()~+~f34()~\},}{\scriptsize\par}

\textcolor{blue}{\scriptsize{}~~go~\{~case~h5(\_)~+~f51(\_)~+~f45(\_)~$\Rightarrow$~eat(5);~t5()~+~f51()~+~f45()~\}}{\scriptsize\par}

\textcolor{blue}{\scriptsize{})}{\scriptsize\par}

\textcolor{blue}{\scriptsize{}t1()~+~t2()~+~t3()~+~t4()~+~t5()}{\scriptsize\par}

\textcolor{blue}{\scriptsize{}f12()~+~f23()~+~f34()~+~f45()~+~f51()}~\\
\end{lyxcode}
Source code: \href{https://github.com/Chymyst/jc-talk-2017-examples/blob/master/src/main/scala/io/chymyst/talk_examples/DiningPhilosophers.scala}{DiningPhilosophers.scala}

For more examples, see the \href{https://github.com/Chymyst/chymyst-core}{code repository}
(first-of, barriers, rendezvous, critical sections, readers/writers,
Game of Life, 8 queens, etc.)
\end{frame}

\begin{frame}{Reasoning about code in the Chemical Machine paradigm}

Chemical metaphor vs.~concurrent data metaphor:
\begin{itemize}
\item Emit molecule with value $\approx$ lift data into the ``concurrent
world''
\item Define reaction $\approx$ lift a function into the ``concurrent
world''
\item Reaction site $\approx$ container for concurrent functions and data
items
\item Reaction consumes molecules $\approx$ function consumes input values
\item Reaction emits molecules $\approx$ function returns result values
\end{itemize}
Reasoning about code:
\begin{itemize}
\item What data do we need to handle concurrently? (Put it on molecules.)
\item What computations consume this data? (Define reactions.)
\end{itemize}
Guarantees:
\begin{itemize}
\item Molecule emitters and reactions are immutable values in local scopes
\item Reaction sites are immutable once activated; can refactor to libraries
\item Molecules are consumed atomically by reactions
\end{itemize}
\end{frame}

\begin{frame}{Current features of \texttt{Chymyst}}

\begin{itemize}
\item Blocking molecules with timeouts
\item Automatic pipelining of molecules
\item ``Static'' molecules with read-only access (similar to Akka ``agents'')
\item Compile-time and early run-time DSL error reporting
\item Logging, debugging, unit-testing facilities
\item Thread pools with thread priority control
\end{itemize}
\end{frame}

\begin{frame}{Related frameworks: Petri nets}

Workflow management with an approach based on \href{https://en.wikipedia.org/wiki/Petri_net}{Petri nets}
\begin{itemize}
\item \href{https://github.com/ing-bank/baker}{ING Baker} -- a DSL for
workflow management
\item Process modeling and control (``elevator system'' etc.)
\item Business process management (BPM) systems
\end{itemize}
\texttt{Chymyst} implements a rich version of Petri nets:
\begin{itemize}
\item Transitions admit arbitrary guard conditions and error recovery
\item Transitions carry values, reactions are values, can be nested
\item Nondeterministic, asynchronous, parallel execution
\end{itemize}
\end{frame}

\begin{frame}{Distributed Chemical Machine}


\framesubtitle{Run concurrent code on a cluster with no code changes}
\begin{itemize}
\item \vspace{-0.2cm}
Same as CM except some molecules are declared as ``distributed''
\item No other code changes necessary!
\begin{itemize}
\item early prototype in progress
\end{itemize}
\end{itemize}
A simple implementation of map/reduce in DCM:
\begin{lyxcode}
\textcolor{blue}{\scriptsize{}implicit~val~cluster~=~ClusterConfig(???)}{\scriptsize\par}

\textcolor{blue}{\scriptsize{}val~c~=~dm{[}Int{]}~;~val~d~=~dm{[}Int{]}}\textcolor{gray}{\scriptsize{}~//~distributed}{\scriptsize\par}

\textcolor{blue}{\scriptsize{}val~res~=~m{[}(Int,~List{[}Int{]}){]}}{\scriptsize\par}

\textcolor{blue}{\scriptsize{}val~fetch~=~b{[}Unit,~List{[}Int{]}{]}}{\scriptsize\par}

\textcolor{blue}{\scriptsize{}site(}{\scriptsize\par}

\textcolor{blue}{\scriptsize{}~~go~\{~case~c(x)~$\Rightarrow$~d(x~{*}~2)~\},}\textcolor{gray}{\scriptsize{}~~//~``map''~on~cluster,}\textcolor{blue}{\scriptsize{}~}{\scriptsize\par}

\textcolor{blue}{\scriptsize{}~}\textcolor{gray}{\scriptsize{}//~``reduce''~on~the~driver~node~only}{\scriptsize\par}

\textcolor{blue}{\scriptsize{}~~go~\{~case~res((n,~list))~+~d(x)~$\Rightarrow$~res((n-1,~s::list))~\},}{\scriptsize\par}

\textcolor{blue}{\scriptsize{}~}\textcolor{gray}{\scriptsize{}//~fetch~results}{\scriptsize\par}

\textcolor{blue}{\scriptsize{}~~go~\{~case~fetch(\_,~reply)~+~res((0,~list))~$\Rightarrow$~reply(list)~\}}{\scriptsize\par}

\textcolor{blue}{\scriptsize{})}{\scriptsize\par}

\textcolor{blue}{\scriptsize{}if~(isDriver)~\{}\textcolor{gray}{\scriptsize{}~//~`true`~only~on~the~driver~node}\textcolor{blue}{\scriptsize{}~}{\scriptsize\par}

\textcolor{blue}{\scriptsize{}~~Seq(1,~2,~3).foreach(x~$\Rightarrow$~c(x))}{\scriptsize\par}

\textcolor{blue}{\scriptsize{}~~res((3,~Nil))~;~fetch()}\textcolor{gray}{\scriptsize{}~//~Returns~the~result.}{\scriptsize\par}

\textcolor{blue}{\scriptsize{}\}}{\scriptsize\par}
\end{lyxcode}
Comparison: \href{https://github.com/ltronky/MapReduce-akka}{Akka implementation of distributed map/reduce}
(400+ LOC)
\end{frame}

\begin{frame}{Reasoning in the Distributed Chemical Machine}

Distributed computing is made declarative
\begin{itemize}
\item Determine which data needs to be distributed and/or concurrent
\item Determine which computations will need to consume that data
\item Emit initial molecules and let the DCM run
\end{itemize}
Peer-to-peer architecture
\begin{itemize}
\item All DCM peers operate in the same way (no master/worker)
\item All DCM peers need to define the same distributed reaction sites
\begin{itemize}
\item To designate a DCM peer as a ``driver'', use config files
\end{itemize}
\item Distributed molecules may be consumed by \emph{any} DCM peer
\end{itemize}
Examples (see documentation)
\begin{itemize}
\item Broadcast (DCM peers see it exactly once upon connecting)
\item Distributed peer-to-peer chat
\end{itemize}
\end{frame}

\begin{frame}{Chemical Machine: implementation details}

\begin{itemize}
\item Each reaction site has a scheduler thread and a worker thread pool
\item Each molecule is bound to a unique reaction site and is stored there
\item Each emitted molecule triggers a search for possible reactions
\item Reactions are scheduled on the worker thread pool
\item Each reaction may emit further molecules
\item Scala macros are used for static analysis and optimizations
\begin{itemize}
\item Automatically pipelined molecules
\item Simplify and analyze Boolean conditions
\end{itemize}
\item Some error analysis is performed at early run time
\end{itemize}
\end{frame}

\begin{frame}{Distributed Chemical Machine: implementation details}

\begin{itemize}
\item Each distributed molecule (DM) is bound to a unique reaction site
\item Emitted DM data goes into the ZK instance
\item Each DCM peer listens to ZK messages and checks for DMs
\item On a DCM peer, each DM is identified with a unique local RS
\begin{itemize}
\item In this way, downloaded molecules can be emitted locally
\item All DCM peers must run identical reaction code 
\end{itemize}
\item Each DCM peer acquires a distributed lock on its DMs
\item If a node goes down or network fails, molecules will be \emph{unconsumed}
\begin{itemize}
\item Another DCM peer will pick up these molecules later
\end{itemize}
\end{itemize}
\end{frame}

\begin{frame}{Conclusions and outlook}

\begin{itemize}
\item Chemical machine = declarative, purely functional concurrency
\begin{itemize}
\item Similar to ``Actors'', but easier to use and ``more purely functional''
\item Short, declarative code implementing barriers, rendezvous, etc.
\end{itemize}
\item An open-source Scala implementation: \texttt{\href{https://github.com/Chymyst/chymyst-core}{Chymyst}}
\begin{itemize}
\item Static DSL code analysis (with Scala macros)
\item Industry-strength features (thread priority control, pipelining, fault
tolerance, unit testing and debugging APIs)
\item Extensive documentation: \href{https://winitzki.gitbooks.io/concurrency-in-reactions-declarative-multicore-in/content/}{tutorial book}
and \href{https://github.com/winitzki/talks/blob/master/join-calculus-paper/join-calculus-paper.pdf}{draft paper}
\end{itemize}
\item Promising applications:
\begin{itemize}
\item Workflow management
\item Distributed peer-to-peer systems
\item Process modeling, GUIs, BPM
\end{itemize}
\item Example code for \href{https://github.com/winitzki/talks/blob/master/join_calculus/join_calculus_2017_SBTB.pdf}{this talk}:
{\footnotesize{}\href{https://github.com/Chymyst/jc-talk-2017-examples}{github.com/Chymyst/jc-talk-2017-examples}}{\footnotesize\par}
\end{itemize}
\end{frame}

\end{document}
}{github.com/Chymyst/jc-talk-2017-examples}}{\footnotesize\par}
\end{itemize}
\end{frame}

\end{document}
