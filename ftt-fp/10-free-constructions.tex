\batchmode
\makeatletter
\def\input@path{{/Users/sergei.winitzki/Code/talks/ftt-fp/}}
\makeatother
\documentclass[english,,russian]{beamer}
\usepackage[T2A,T1]{fontenc}
\usepackage[utf8]{inputenc}
\setcounter{secnumdepth}{3}
\setcounter{tocdepth}{3}
\usepackage{babel}
\usepackage{tipa}
\usepackage{tipx}
\usepackage{amsmath}
\ifx\hypersetup\undefined
  \AtBeginDocument{%
    \hypersetup{unicode=true,pdfusetitle,
 bookmarks=true,bookmarksnumbered=false,bookmarksopen=false,
 breaklinks=false,pdfborder={0 0 1},backref=false,colorlinks=true}
  }
\else
  \hypersetup{unicode=true,pdfusetitle,
 bookmarks=true,bookmarksnumbered=false,bookmarksopen=false,
 breaklinks=false,pdfborder={0 0 1},backref=false,colorlinks=true}
\fi

\makeatletter

%%%%%%%%%%%%%%%%%%%%%%%%%%%%%% LyX specific LaTeX commands.
\DeclareRobustCommand{\cyrtext}{%
  \fontencoding{T2A}\selectfont\def\encodingdefault{T2A}}
\DeclareRobustCommand{\textcyr}[1]{\leavevmode{\cyrtext #1}}


%%%%%%%%%%%%%%%%%%%%%%%%%%%%%% Textclass specific LaTeX commands.
% this default might be overridden by plain title style
\newcommand\makebeamertitle{\frame{\maketitle}}%
% (ERT) argument for the TOC
\AtBeginDocument{%
  \let\origtableofcontents=\tableofcontents
  \def\tableofcontents{\@ifnextchar[{\origtableofcontents}{\gobbletableofcontents}}
  \def\gobbletableofcontents#1{\origtableofcontents}
}
\newenvironment{lyxcode}
  {\par\begin{list}{}{
    \setlength{\rightmargin}{\leftmargin}
    \setlength{\listparindent}{0pt}% needed for AMS classes
    \raggedright
    \setlength{\itemsep}{0pt}
    \setlength{\parsep}{0pt}
    \normalfont\ttfamily}%
   \def\{{\char`\{}
   \def\}{\char`\}}
   \def\textasciitilde{\char`\~}
   \item[]}
  {\end{list}}

%%%%%%%%%%%%%%%%%%%%%%%%%%%%%% User specified LaTeX commands.
\usetheme[secheader]{Boadilla}
\usecolortheme{seahorse}
\title[Chapter 10: Free type constructions]{Chapter 10: Free type constructions}
%\subtitle{Part 2: Their laws and structure}
\author{Sergei Winitzki}
\date{2018-11-22}
\institute[ABTB]{Academy by the Bay}
\setbeamertemplate{headline}{} % disable headline at top
\setbeamertemplate{navigation symbols}{} % disable navigation bar at bottom
\usepackage[all]{xy}
\usepackage[nocenter]{qtree}
\makeatletter
% Macros to assist LyX with XYpic when using scaling.
\newcommand{\xyScaleX}[1]{%
\makeatletter
\xydef@\xymatrixcolsep@{#1}
\makeatother
} % end of \xyScaleX
\makeatletter
\newcommand{\xyScaleY}[1]{%
\makeatletter
\xydef@\xymatrixrowsep@{#1}
\makeatother
} % end of \xyScaleY
\newcommand{\shui}{\begin{CJK}{UTF8}{gbsn}水\end{CJK}}
\usepackage{CJKutf8} % For occasional Chinese characters. Also, add "russian" to documentclass.

\makeatother

\begin{document}
\frame{\titlepage}
\begin{frame}{The interpreter pattern I. Expression trees as programs}

Represent a program as a data structure, run later
\begin{itemize}
\item Example: a simple DSL for complex numbers
\end{itemize}
\begin{minipage}[t]{0.35\columnwidth}%
\begin{lyxcode}
\textcolor{blue}{\footnotesize{}val~a~=~\textquotedbl 1+2{*}i\textquotedbl .cplx}{\footnotesize\par}

\textcolor{blue}{\footnotesize{}val~b~=~a~{*}~\textquotedbl 3-4{*}i\textquotedbl .cplx}{\footnotesize\par}

\textcolor{blue}{\footnotesize{}b.conj}{\footnotesize\par}
\end{lyxcode}
%
\end{minipage}\qquad{}\qquad{}%
\begin{minipage}[t]{0.45\columnwidth}%
\begin{lyxcode}
\textcolor{blue}{\footnotesize{}Conj(}{\footnotesize\par}

\textcolor{blue}{\footnotesize{}~~Mul(}{\footnotesize\par}

\textcolor{blue}{\footnotesize{}~~~Cplx(\textquotedbl 1+2{*}i\textquotedbl ),~Cplx(\textquotedbl 3-4{*}i\textquotedbl )}{\footnotesize\par}

\textcolor{blue}{\footnotesize{}))}{\footnotesize\par}
\end{lyxcode}
%
\end{minipage}
\begin{itemize}
\item \texttt{\textcolor{blue}{\footnotesize{}Cplx}}, \texttt{\textcolor{blue}{\footnotesize{}Mul}},
\texttt{\textcolor{blue}{\footnotesize{}Conj}} etc.~ must be linear and distributive (but not commutative):
\begin{align*}
\mathbf{u}_{1}\otimes\mathbf{v}_{1}+\left(\mathbf{u}_{2}\otimes\mathbf{v}_{2}+\mathbf{u}_{3}\otimes\mathbf{v}_{3}\right) & =\left(\mathbf{u}_{1}\otimes\mathbf{v}_{1}+\mathbf{u}_{2}\otimes\mathbf{v}_{2}\right)+\mathbf{u}_{3}\otimes\mathbf{v}_{3}\\
\mathbf{u}\otimes\left(a_{1}\mathbf{v}_{1}+a_{2}\mathbf{v}_{2}\right) & =a_{1}\left(\mathbf{u}\otimes\mathbf{v}_{1}\right)+a_{2}\left(\mathbf{u}\otimes\mathbf{v}_{2}\right)\\
\left(a_{1}\mathbf{v}_{1}+a_{2}\mathbf{v}_{2}\right) & \otimes\mathbf{u}=a_{1}\left(\mathbf{v}_{1}\otimes\mathbf{u}\right)+a_{2}\left(\mathbf{v}_{2}\otimes\mathbf{u}\right)
\end{align*}

\begin{itemize}
\item We have such a product for 3-dimensional vectors only; ignore that
\end{itemize}
\item Consider \emph{unevaluated} \emph{expressions} of the form $\mathbf{u}_{1}\otimes\mathbf{v}_{1}+\mathbf{u}_{2}\otimes\mathbf{v}_{2}+...$
\begin{itemize}
\item A free vector space generated by pairs of vectors
\end{itemize}
\item Impose the equivalence relationships shown above
\begin{itemize}
\item The result is known as the \textbf{tensor product}
\end{itemize}
\item Redundant encoding: unevaluated expression tree
\begin{itemize}
\item A list of any number of vector pairs $\sum_{i}\mathbf{u}_{i}\otimes\mathbf{v}_{i}$
\end{itemize}
\item Reduced encoding: a matrix
\begin{itemize}
\item Reduced encoding requires proof and more complex operations
\end{itemize}
\end{itemize}
\end{frame}

\begin{frame}{Exercises}
\begin{enumerate}
\item {\footnotesize{}\vspace{-0.15cm}Show that }{\footnotesize\par}
\end{enumerate}
\end{frame}

\end{document}
